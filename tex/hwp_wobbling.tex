\subsection{Polarization Modulators/Wobbling}

\paragraph{Description:}
By wobbling we mean a non uniform HWP mechanical rotation. In standard HWP rotation, the HWP disc remains parallel to the plane 
of its own rotor. If the HWP wobbles, the HWP disc is not anymore parallel. For semplicity we can think it changes with time describing
a sinusoidal law. The value of the amplitude and on the phase depends on the HWP rotators and its own imperfections,  it cannot be derived 
from analytical considerations. A simulation can only provide how the wobbling can affect the HWP signal. 

\paragraph{Plan to model and/or measure:}
The best way to characterize the HWP wobbling is measuring it by putting some reflecting material on top of the rotating HWP. 
Sideband of 2f or 4f HWP synchronous signal
can be used as well.

\paragraph{Uncertainty/Range:}
Uncertainty could come from a too much naive choice of the sinusoidal law, since it might
might be not applicable or it might change with time. The uncertainty in the estimate of this effect is ultimately 
related to the HWP angle encoder resolution.

\paragraph{Parameterization:}
The most naive way to parameterize the HWP wobbling is assuming an incident angle varying with a co-sinusoidal law.
Since the normal to the HWP and the disc are perpendicular each other, the variation with time of the HWP disc can be translated
into variation of the incident angle. Since is highly unprobable that a possible wobbling effect will keep the same amplitude,
a more realistic parameterization of this effect should consider random angle fluctuations.

SRF: 3(2)*
