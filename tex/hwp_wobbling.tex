\subsection{Wobbling}

\paragraph{Description:}
A non uniform HWP mechanical rotation can impact the output signal.
By wobbling we mean a non uniform HWP mechanical rotation. The simplest not uniform rotation is the wobbling, which
we define as a slight change of the HWP rotation axis happening during the HWP rotation itself. In standard HWP rotation, the HWP disc remains parallel to the plane
of its own rotor. If the HWP wobbles, the HWP disc is not anymore parallel.

The reasons can be typically of mechanical origin,
e.g.: excessive play in the HWP mounting support, erroneous HWP mounting in its own support. The wobbling shouldn't be confused with the vibration. By vibration we mean random $a$ movements of the HWP (with $a \ll d$, being $d$ the HWP thickness) in all the directions; with wobbling, instead, we mean random movements of the HWP generated by a precession of the HWP rotation axis. Since different mechanical mismatches are prone to create a wobbling there is not a unique way to model it; therefore, in what follows we have modeled it assuming a simple test model.

For semplicity we can think it changes with time describing
a sinusoidal law. The value of the amplitude and on the phase depends on the HWP rotators and its own imperfections, it cannot be derived
from analytical considerations. A simulation can only provide how a typical wobbling affects the HWP output signal.

Results from the 2$f$ signal generated by a HWP wobbling with the law $5\cos(i)$ (Fig.\,\ref{hwpwobble}), with $i$ the incident angle of the incoming radiation, shows a modest impact on the signal of interest.

\paragraph{Plan to model and/or measure:}
The easiest way to model a HWP wobbling is varying, in a random way, the incident angle of the incoming radiation crossing the HWP.
In case of a macroscopic movement, the effect can be studied just optically observing the behavior of a
reflecting material applied on a rotating HWP. A more quantitative way would require studying the
sideband of 2$f$ or 4$f$ HWP synchronous signal.


\paragraph{Uncertainty/Range:}
Uncertainty could come from a too much naive choice of the sinusoidal law: a simple, and unique, co-sinusoidal law might be not applicable or it might change with time, i.e. it could be described by different co-sinusoidal laws. The uncertainty in the estimate of this effect is ultimately
related to the HWP angle encoder resolution.

\paragraph{Parameterization:}
The most naive way to parameterize the HWP wobbling is assuming an incident angle varying with a co-sinusoidal law.
Since the normal to the HWP and the disc are perpendicular each other, the variation with time of the HWP disc can be translated
into variation of the incident angle. Since is highly unlikely that a wobbling effect will keep the same amplitude across time,
a more realistic parameterization of this effect should consider injecting random angle fluctuations as well.

%SRF: 3(2)*

%\begin{figure}
%\centering
%\includegraphics[width=2.5in]{figures/hwp_wobble.png}
%\caption 2$f$ signal from a (1,0,0,0) Stokes vector incident on a Sapphire HWP wobbling with the incoming radiation varying with the law $5\cos(i)$ with $i$ the incident angle of the incoming radiation.}
%\label{hwpwobble}
%\end{figure}
