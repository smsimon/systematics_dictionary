
\subsection{HWP Beam Systematics}


\paragraph{Description:}
Having a HWP skyward of every optical element mitigates a range of beam and polarization systematics. However, when the HWP is further down the optical chain as in the case of a cryogenic HWP, detailed modeling is necessary to determine what systematics it induces and mitigates.

\textbf{Do we have examples of what beam properties get worse?}

\paragraph{Plan to model and/or measure:}
Modeling would require the HWP optical properties and both ray tracing and physical optics for several cases, including non-planar positions and non-uniform AR coatings. The HWP would need to be included in full optics simulations of the optics chain. Commercially available software like CST can model diffraction effects, in a reasonable CPU time, for complicated, large structures, so a model could be developed for the HWPs and serve as input to the larger optical studies.

\textbf{Is there more info on how we can model/measure/calibrate/mitigate this effect? How do we account for mitigation and effects when it is included in the full optics configuration}

Beam maps with and without the HWP can also help determine the HWP's impact on the beams. Measurements, in principle, would be done with a long campaign of beam maps taken at a large set of HWP angle, and/or, in the case of a rapid modulator, very slow beam maps taken with several modulation cycles per beam scanned (differently from sky configuration where the telescope scan is faster). Both unpolarized and polarized beam maps would be necessary.

Optical effects of the HWP must be modeled as there is a large uncertainty in the size of the effect, so the SRF is 5.

\paragraph{Uncertainty/Range:}
\textbf{Do we have any modeled values from SO?}

\paragraph{Parameterization:}
This can be expressed as the beam with and without the HWP from optics simulations.
