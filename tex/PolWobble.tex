\subsection{Polarization Wobble : Sinuous Antenna}

A sinuous antenna is part of a class of planar antenna geometries called log-periodic antennas due to a log-periodic winding of the antenna arms. The benefit of log periodic antennas is the fact that their properties such as impedance, and beam properties stay consistent over a wide bandwidth and repeat every $ln(\tau)$ where $\tau$ is the characteristic length scale over which the antenna arm pattern repeats. The maximum and minimum bandwidth of these antennas are in theory just set by their inner and outer radii. One characteristic of log periodic antennas that is problematic for polarization measurements of the CMB is polarization wobble. This is a rotation of the polarization axis that oscillates between a maximum and minimum value every $\ln(\tau)$. Within the class of log-periodic antennas the sinuous antenna has the lowest amount of polarization wobble and the amount of antenna wobble is set by $\tau$: lower $\tau$ corresponds to smaller max to min deviation in polarization axis. Detailed studies of the PB2 sinuous antennas can be found in \cite{Obrient2008},\cite{Edwards2012}.


Polarization wobble is primarily a problem because is creates cross polarization or Q to U leakage, which translates into E to B leakage. The wobble can additionally induce temperature to B-mode leakage from the ellipticity angle not alligning with the orthogonal directions of the polarization axis. Systematics from PB2 antennas as well as detectors and lenslets can be found in some details in \cite{TokiThesis}.

\paragraph{Plan to model and/or measure:}
Minimizing $\tau$ minimizes cross polarization but smaller $\tau$ also sets the required linewidths of your traces. For POLARBEAR a $\tau$ of 1.3 was chosen to minimize $\tau$ but still allow for fabrication constraints on trace widths. To further null the effect of cross polarization from polarization wobble a 4 pixel differencing scheme has been proposed \cite{TokiThesis},\cite{TokiMemo1},\cite{TokiMemo2}, which will return Q and U values independent of the polarization wobble angle. This procedure requires two sets of two pixels each (A and B as shown in Fig.~1) with each set having one antenna with one linear polarization oriented at 0 degrees and the second pixel rotated by 45 degrees. The polarization wobble angle is denoted as $\phi(\nu)$, the angle of incident light is $\theta(\nu)$, the detectors efficiency is $\eta(\nu)$, and the electric field amplitude is $E(\nu)$. This setup is depicted in Figure~1. The steps to extract Q and U values for the 4 pixels with the dependence on $\phi(\nu)$ removed is described below. The relative power on the the 90 and 45 pixels of the A and B sets is given by
\begin{figure}
\centering
\includegraphics[width=2.5in]{figures/4pixelremovewobble.png}
\caption{Illustration of 4 pixels with polarization wobble. Each set A and B have both a 0 \& 90 degree orientation antenna and a 45 \& -45 degree orientation pixel. $\theta$ is the polarization angle of incoming light, and $\phi$ is the polarization wobble angle.}
\label{4pixelwobbleremoval}
\end{figure}
\begin{equation}
\begin{split}
&P_{A0} = \int \eta(\nu)[E(\nu)cos(\theta(\nu)-\phi(\nu))]^2 d\nu \\
&P_{A90} = \int \eta(\nu)[E(\nu)sin(\theta(\nu)-\phi(\nu))]^2 d\nu \\
&P_{A45} = \int \eta(\nu)[E(\nu)cos(\frac{\pi}{4}-\theta(\nu)+\phi(\nu))]^2 d\nu \\
&P_{A-45} = \int \eta(\nu)[E(\nu)sin(\frac{\pi}{4}-\theta(\nu)+\phi(\nu))]^2 d\nu \\
&P_{B0} = \int \eta(\nu)[E(\nu)cos(\theta(\nu)+\phi(\nu))]^2 d\nu \\
&P_{B90} = \int \eta(\nu)[E(\nu)sin(\theta(\nu)+\phi(\nu))]^2 d\nu \\
&P_{B45} = \int \eta(\nu)[E(\nu)cos(\frac{\pi}{4}-\theta(\nu)-\phi(\nu))]^2 d\nu \\
&P_{B-45} = \int \eta(\nu)[E(\nu)sin(\frac{\pi}{4}-\theta(\nu)-\phi(\nu))]^2 d\nu .
\end{split}
\end{equation}
We then assume that $\theta$ is constant across our spectral band and difference opposite orientation detectors to extract Q and U:
\begin{equation}
\begin{split}
&Q_A= P_{A0}-P_{A90}=\int \eta(\nu)E^2(\nu)cos[2(\theta-\phi(\nu))]d\nu \\
&Q_B = P_{B0}-P_{B90}=\int \eta(\nu)E^2(\nu)cos[2(\theta+\phi(\nu))] d\nu \\
&U_A = P_{A45}-P_{A-45}\int \eta(\nu)E^2(\nu)sin[2(\theta-\phi(\nu))] d\nu \\
&U_B =  P_{B45}-P_{B-45}\int \eta(\nu)E^2(\nu)sin[2(\theta+\phi(\nu))] d\nu \\
\end{split}
\end{equation}
Using the trig angle sum formula, we can expand this to
\begin{equation}
\begin{split}
&Q_A= \int \eta(\nu)E^2(\nu)[cos(2(\theta)cos(2\phi(\nu))+sin(2(\theta)sin(2\phi(\nu))]d\nu \\
&Q_B = \int \eta(\nu)E^2(\nu)[cos(2(\theta)cos(2\phi(\nu))-sin(2(\theta)sin(2\phi(\nu))]d\nu \\
&U_A = \int \eta(\nu)E^2(\nu)[sin(2(\theta)cos(2\phi(\nu))-cos(2(\theta)sin(2\phi(\nu))]d\nu \\
&U_B =  \int \eta(\nu)E^2(\nu)[sin(2(\theta)cos(2\phi(\nu))+cos(2(\theta)sin(2\phi(\nu))]d\nu . \\
\end{split}
\end{equation}
Taking linear combinations of the above Q's and U's we can define new Q's and U's as
\begin{equation}
\begin{split}
&Q_1=\frac{Q_A+Q_B}{2}= cos(2\theta)\int \eta(\nu)E^2(\nu)cos(2\phi(\nu))d\nu \\
&Q_2 = \frac{U_B-Q_A}{2}= cos(2\theta)\int \eta(\nu)E^2(\nu)sin(2\phi(\nu))d\nu \\
&U_1 = \frac{Q_A-Q_B}{2}= sin(2\theta)\int \eta(\nu)E^2(\nu)sin(2\phi(\nu))d\nu \\
&U_2 =  frac{U_A+U_B}{2}= sin(2\theta)\int \eta(\nu)E^2(\nu)cos(2\phi(\nu))d\nu. \\
\end{split}
\end{equation}
You can then extract $\theta$ as
\begin{equation}
\theta=\frac{1}{2}tan^{-1}\frac{U_{1,2}}{Q_{2,1}}.
\end{equation}
The E-field amplitude can also be determined form this scheme as:
\begin{equation}
E^2=\frac{P_{A0}+P_{A90}}{\int\eta(\nu)d\nu}=\frac{P_{A45}+P_{A-45}}{\int\eta(\nu)d\nu}=\frac{P_{B0}+P_{B90}}{\int\eta(\nu)d\nu}=\frac{P_{B45}+P_{B-45}}{\int\eta(\nu)d\nu}
\end{equation}
Where $\eta(\nu)$ is measured using an FTS. 

The polarization wobble is a large effect that causes significant leakage into the B-mode spectrum, but it is well-understood. It thus needs to be modeled and mitigated, making its SRF a 4.

\paragraph{Uncertainty/Range:}
This analysis assumes that all channels have the same response $\eta(\nu)$, but there is some variation across detectors, so uniformity across the array can limit the effectiveness of this method. This method also requires accurate FTS measurements of the spectral response. It also assumes that the beams are perfectly symmetric, so further uncertainty can be introduced from ellipitical beams. For a trypical sinuous detector, the ellipticity is small with a level of $<2$\% and varies with frequency. \textbf{What are additional sources of uncertainty?}

\textbf{How much suppression of the E to B leakage does this method provide? How much suppression of the T to B leakage does it provide?}

\paragraph{Parameterization:}
The leakage from the polarization wobble can be parameterized as the contaminated Q and U maps to estimate the power spectra leakages. Using the four pixel subtraction method in map space can determine how well these leakages can be mitigated.

\textbf{Make sure these commented out references are in the systematics dictionary references}
%\subsubsection{References}
%\begin{itemize}
%\item O'Brient, B. \textit{et.al.}, "Sinuous-Antenna coupled TES bolometers for Cosmic Microwave Background
%Polarimetry," LTD13 Proceedings (2009).
%\item Edwards, J.M. \textit{et.al.}, "Dual-Polarized Sinuous Antennas on Extended
%Hemispherical Silicon Lenses," IEEE Trans Antennas and Propagation, Vol 60, No 9 (2012).
%\item Suzuki, A. "Lenslet Coupled Sinuous Antenna Systematic Study", Berkeley Memo (2015). - Ask for this, its not published.
%\item Suzuki, A. "Sinuous Antenna Wobble Cancellation", Berkeley Memo (2012). - Ask for this, its not published.
%\item O'Brient, B. \textit{et.al.}, "Sinuous Antennas for Cosmic Microwave Background
%Polarimetry," SPIE, Vol 7020 70201H-1 (2008).
%\item Suzuki, A. "Multichroic Bolometric Detector Architecture for Cosmic Microwave Background Polarimetry Experiments," Berkeley Dissertation (2013).
%\end{itemize} 
