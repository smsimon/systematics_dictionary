\subsection{Pointing}

\paragraph{Description:}
Pointing reconstruction is necessary in order accurate know where the instrument is looking at on the sky at any particular time. It is well known that many mechanical, structural, and environmental factors will affect the telescope pointing on the sky and a pointing model is needed in order to recover the pointing accuracy to the level needed for high ell science. The pointing systematics can be categorized into random pointing jitter, systematic pointing error, and optical pointing distortions.

Pointing models are very commonly used for astronomical telescopes in order to reconstruct accurate telescope pointing. Commonly used pointing model parameters are described in Mangum, 2001. Typically structural deformations and tilts, encoder offsets, and timing errors are taken into account through the pointing model. Pointing models are calculated through dedicated pointing observations of point-like sources across the sky in azimuth and elevation. Ideally uniform sampling across the sky is necessary but in many cases bright point-like sources are not available in various parts of the sky which can lead to errors in the pointing reconstruction.

Random pointing jitter is the statistical RMS error in the pointing reconstruction. In the case that the pointing error is random, this error can be modeling as a blurring or broadening of the telescope beam. Hence the effect is equivalent to decreased sky resolution. This can be modeled in terms of $B_{\ell}$
\begin{equation}
B_{\ell}^{eff} = B_{\ell} e^{\frac{-\ell(\ell+1)}{2} \sigma^{2}_{p,RMS}}
\end{equation}
where $\sigma^{2}_{p,RMS}$ is the RMS pointing error. 

Systematic pointing error is the residual error in the pointing reconstruction dependent on external parameters. This pointing error is due to optical elements moving relative to each other causing optical misalignment. For example it is well known that thermal and solar heating and cooling of the telescope can cause large amounts of pointing error on the orders of 10s of arcseconds. An overall ambient temperature change can cause pointing drifts across time. Furthermore differential solar heating creating temperature gradients across the telescope structure can cause complex pointing changes that depend on the structural shape of the telescope itself. This error is difficult to model due to its complex dependence on telescope orientation and environmental parameters
\begin{equation}
\sigma_{p,sys} \left ( Az, El, T, I_{R}, v_{w}, ... \right )
\end{equation}
In the case that there is uniform sampling across each of these parameters and the errors are small relative to the beam size, then the pointing error can be roughly approximated similar to the pointing jitter, but this may not always be the case. If the systematic nature is complex, then the effective beam may become non-Gaussian in ways difficult to model.

Optical pointing distortions are pointing errors due to deformation of the optical elements themselves. This error is distinguished from systematic pointing errors above in that the optics itself changes rather than just alignment. For example the primary mirror can deform due to solar heating during the daytime and change the F number of the system. Pointing distortions are harder to correct than systematic pointing error because they are dependent on the tolerance of the optical system and hence affect the optical performance of the telescope. Typically the effects are larger in degree than the other pointing errors and may only be correctable by active realignment of the telescope itself or insulating optical elements well.
\begin{equation}
\delta_{p} \left ( x, y, M, ... \right )
\end{equation}
Here $M$ represents the Mueller matrix beam model calculable through physical optics simulations. 

\paragraph{Plan to model and/or measure:}
Frequent and regular observations of bright point-like sources across the sky are needed in order to calculate a pointing model. With good coverage across azimuth and elevation, the pointing jitter can be minimized well. With good coverage the jitter error will eventually be limited by errors in the analysis itself such as fitting for beam centroids.

This pointing model can be expanded to include parameters that model environmental effects as done in POLARBEAR. This would allow for modeling and minimizing the systematic error, but there is a limit to how well these effects can be corrected using the pointing model alone. Regular observations around the year as well as at various times during the day will be needed to assess the impact of this error and its dependence to the environmental parameters.

Optical pointing distortions need to be mainly modeled using physical optics or structural analysis of the optical elements under various environmental conditions. This is by far the hardest to model and measure.

\paragraph{Uncertainty/Range:}
This section should include the uncertainty of
known parameters and/or the expected range of parameters for consideration

\paragraph{Parameterization:}
This section should include the parameterization of figures of
merit and the output to the SWGs.
