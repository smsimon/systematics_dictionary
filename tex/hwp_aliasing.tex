\subsection{HWP Aliasing}

\paragraph{Description:}

A continuously rotating HWP modulates polarization and shifts the frequency range of large angular scale polarization signals to higher temporal frequencies in the time ordered data. However, this separation of temperature and polarization in temporal frequency is imperfect. In general, there is still a signal from the (beam convolved) small scale temperature anisotropies at the modulation frequency of the HWP appears as a polarization signal. (Here small angular scales is taken to mean the spatial frequency corresponding to the difference between $0f$ and $4f$). Since this effect directly mixes the high frequency content of the temperature signal into the low frequency recovered polarization we call this \emph{HWP aliasing}.

Additionally, the temperature signal can appear in the sidebands of HWP harmonics other than the $4f$. These cause a similar aliasing effect to the temperature signal itself, however the temperature angular scale now corresponds to the difference in temporal frequency between the harmonic and $4f$. This version of the effect is suppressed by the leakage coefficient for each harmonic. In particular, the $2f$ may be problematic because the $2f$ tends to be larger than the other harmonics and there is a known temperature leakage as large as $\sim 1\%$ from the differential transmission of the HWP. Aliasing from the $3f$ and $5f$ due to the HWP non-uniformity may also concerning since these harmonics are close to the modulation frequency.

This effect is naturally mitigated somewhat by the mapmaking operation. Because the $0f$ aliasing signal is unpolarized and the mapmaking operation implicitly differences detector pairs, it can be cancelled to first order by requiring both orthogonal detectors in a pair. Additionally, since the aliasing depends on the half wave plate angle as a function of time, observing the same sky at many HWP angles will cause the effect to converge toward zero in the final map.

\paragraph{Plan to model and/or measure:}

This effect can be directly simulated. The $0f$ aliasing from the CMB (or CMB + foreground) can be simulated by scanning the beam-convolved temperature only map (with sufficient sampling rate) and running the demodulation and mapmaking operations. Alternately, the aliasing effect can be simulated directly in demodulated timestream space by addition of the appropriate detector and HWP angle factors. The aliasing from temperature leakage into HWP harmonics (such as the $2f$) can be simulated in the same way.


%To model this effect we can use the Mueller matrix to estimate the size of the intensity to $2f$ leakage. (More details, how does this give you the aliasing?)
The leakage coefficients for each HWP harmonic can be estimated from an optics simulation or from fitting the correlation between timestreams demodulated around at each HWP harmonic ($2f$, $3f$, $5f$) and the intensity signal $0f$.
This can be done using the same approach used to determine the $4f$ leakage coefficients~\cite{Essinger-Hileman2016,Didier_Thesis,PB1_WHWP}.


This effect directly contaminates the maps in the science frequencies. It is critical that we model it using an accurate scan strategy, beam profile and HWP parameters. This has an SRF of 4, however it may be very strongly suppressed for the reasons given above.

\paragraph{Uncertainty/Range:}
\textbf{Do we have values of how large this is for the SO SAC? If not, do we have sims that can estimate it? How does it vary with frequency and other parameters? Does it vary with HWP position in the instrument?}

\paragraph{Parameterization:}
%This section should include the parameterization of figures of
%merit and the output to the SWGs.
%We can parameterize this effect as an intensity to polarization leakage level in the 4$f$ signal.
The size of this effect is primarily determined by the beam size of the telescope $\sigma$, the scan speed $v$, and the rotation speed of the HWP $\omega$. Larger beam sizes will suppress the small-angular-scale temperature fluctuations that alias into polarization. Faster scan speeds spread the temperature signals to a wider range of temporal frequencies, increasing this effect. Faster HWP rotation makes the HWP harmonics further apart in temporal frequency (and the $4f$ further from $0f$) reducing the effect. A requirement could be written for the combined parameter $\sigma\omega/v$ which would depend on the leakage coefficients.
