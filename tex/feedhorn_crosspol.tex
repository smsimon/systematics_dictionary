\subsection{Cross Polarization from Feedhorn}

\paragraph{Description:}
The cross polarization describes how much leakage there is between orthogonal polarization modes. If not properly accounted for, this can cause leakage between E-modes and B-modes.

\paragraph{Plan to model and/or measure:}
The cross polarization is determined from the polarized ($rEL3X$ and $rEL3Y$) beam parameters in dB from HFSS. The E-plane beam is then given by $rEL3X$ at $\phi=0^{\circ}$, where $\phi$ is the angular coordinate of the beam, and the H-plane beam is given by $rEL3X$ at $\phi=90^{\circ}$. The cross polarization beam is given by $rEL3Y$ at $\phi=45^{\circ}$. All beams are normalized by the maximum value of the E-plane beam, such that the maximum value of the E-plane beam is equal to one. The cross polarization is then the maximum value of the cross polarization beam.

\paragraph{Uncertainty/Range:}
The typical allowable ranges for this value are $<-15$~dB to $<-30$~dB in power, but this needs to be further constrained by the SWG for this specific project. The crosspol for the AdvACT 90/150~GHz feedhorns is $1.74\%$ in the low band and $0.3\%$ in the high band, while the crosspol for the AdvACT 150/230~GHz feedhorns is $1\%$ in the low band and $0.4\%$ in the high band.

\paragraph{Parameterization:}
We can parameterize this effect with the cross-polar beam from HFSS.
