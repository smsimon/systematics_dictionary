% !TEX root =  ../syst_master.tex 

\subsection{$I\rightarrow P$ leakage and Instrumental Polarization Upstream of the HWP}
\label{IP upstream of HWP}

\paragraph{Description:}
Any polarized signal originating from the telescope upstream of the HWP will be rotated and coupled to the detector at 4$f$,
making it indistinguishable from the sky polarization.
This includes the intensity dependent leakage caused by differential transmission and reflection, described and calculated in section \ref{instrumental_polarization} 
along with polarized emission caused by differential absorption.

The transmission, reflection and absorption coefficients of an element must satisfy the relation
\begin{equation}
T + R + A = 1,
\end{equation}
and so the differential coefficients for the $s$ and $p$ polarizations must satisfy
\begin{equation}
T_{s - p} + R_{s-p} + A_{s-p}
= \frac{1}{2}\big[(T_s - T_p)  + (R_s - R_p) + (A_s - A_p)\big] = 0.
\end{equation}

Because the majority of optical elements are operated at low temperatures, 
the power of the polarized emission is usually small compared to the power created by differential transmission.
This is not true for ambient temperature elements, such as the mirrors on the LAT, which generate a large polarized signal. 

\paragraph{Plan to model and/or measure:}
For the majority of optical elements we expect $I\rightarrow P$ leakage to be negligible. The optical elements which dominate the leakage are Alumina filters and the window for the SAT, and mirrors and lenses for the LAT.

On the SAT the beam acts as a plane wave when passing through optical elements, so we are able to calculate the differential coefficients of the Alumina filters and window for varying incident angles using the Transfer Matrix Method.
The differential coefficients are small for detectors at the center of the focal plane, since the incident angle of the beam is small.
For detectors on the edge of the focal plane, the incident angle can be as large as $20^\circ$, so the differential transmission, reflection, and absorption are non-negligible.

Calculation of IP for the LAT is described in section \ref{instrumental_polarization}.

In order to predict the total polarized power generated by an element, we require the unpolarized power incident on both the sky-side and detector-side of the element. We multiply these by the differential transmission and reflection coefficients respectively.
Provided we have the optical chain, we are able to use the transmission, reflection, and absorption coefficients to propagate unpolarized light throughout the system.

To calculate the $A^{(2)}$ and $A^{(4)}$ signal we multiply the stokes vector for the light incident on the HWP with the transmission Mueller matrix for a range of HWP rotation angles, and fit the first eight harmonic amplitudes and phases to the demodulated signal. 
We then multiply by the transmission efficiency of all elements between the HWP and the detector.

\textbf{Say something about the temperature stability of components upstream and how this would be modeled.}

\paragraph{Uncertainty/Range:} \textbf{THESE ARE NOT THE MOST RECENT VALUES. I'll update this soon...}
Using this model for 145 GHz, we can expect values of $A^{(4)}$ in the range of 0.027 pW which converts to 0.399 K$_\text{CMB}$ using the optical chain mentioned in the memo. This increases if the HWP is positioned later in the optical chain because more lenses are taken into account.
\textbf{This needs to be updated for the SAC and include different frequencies.}

If there were a HWP on the LAT where there are ambient temperature mirrors, a majority of this signal ($\sim 80-90\%$) would come from the polarized emission of the mirrors.

\paragraph{Parameterization:}
For this calculation we need to know the properties of the optical chain (e.g. absorption, spill, scatter, reflection coefficients, and 
the temperature of each element chain). In order to use the transfer matrix method we need the complex index of refraction of the optical element and its thickness. This can be parameterized as an overall intensity to polarization leakage.


