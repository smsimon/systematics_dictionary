\subsection{Beam cross-polar response}

\paragraph{Description:}
Cross polarization is an optical systematic that shows how much polarization leakage there is between orthogonal polarization modes. Typically it is a characteristic of the optical design itself and represents how much polarization is rotated as it propagates through the optics. This can be calibrated out with accurate polarization angle calibration but does decrease polarization efficiency. If the calibration is incorrect, this will cause Q and U to leak into each other and cause E modes leaking into B modes. 

Instrumental polarization is an optical systematic that shows how much intensity signal is leaking into the polarization signal. This is also a characteristic of the optical design itself but depends more on the properties of metals and dielectric materials. This systematic will leak the T signal into E and B modes causing large contamination if not accounted for in analysis.

Both systematics can be modeled using the Mueller matrix formalism. If the telescope Mueller beam matrix is known, these systematics (along with beam effects) can be propagated to the Q, U maps by
\begin{equation}
Q' = m_{qq} Q + m_{qu} U, \ \ U' = m_{uu} U + m_{uq} Q
\end{equation}
\begin{equation}
Q' = m_{qq} Q + m_{qi} I, \ \ U' = m_{uu} U + m_{ui} I
\end{equation}
where the first equation is for cross polarization and the second is for instrumental polarization. In this way systematics contaminated Q and U maps can be simulated. The contaminated Q and U maps can be further propagated to the power spectra to see the leakage effect.

\paragraph{Plan to model and/or measure:}
This effect can be modeled using physical optics simulations where the Mueller matrix can be calculated directly. Comparing this to real data may be needed using a polarized source.

\paragraph{Uncertainty/Range:}
Insert text

\paragraph{Parameterization:}
Insert text